\documentclass[12pt]{article}
\usepackage[utf8]{inputenc}
\usepackage{longtable}
\usepackage{color}
\usepackage{multirow}
\usepackage{hyperref}
\usepackage{amssymb}
\usepackage{setspace}
%\doublespacing

\renewcommand*{\sectionautorefname}{Section} 

\title{Predation model \\ Hybridisation\_A} 
\author{Raphael Klein, EPFL}

\usepackage{natbib}
\usepackage{graphicx}
\usepackage[labelfont=bf]{caption} 	% Make captions bold (Figure & Table)
\usepackage{subfig}	
\usepackage{amsmath}
\usepackage{hyperref}
%\usepackage[section]{placeins}

\providecommand{\keywords}[1]{\textbf{Keywords:} #1}

\begin{document}

\maketitle

%\textcolor{red.green.blue.cyan.yellow.magenta.}{}

This reports the hybridisation of the predation model with the ACF implementation of the policy model. It outlines how the models were hybridised and goes through the initialisation of the models and the experiments that will be run using these models.

%%%%%%%%%%%%%%%%%%%%%%%%%%%%%%%%%%%%%%%%%%%%%%%%%%%%%%%%%%%%%%%%%%
\section{The problem tree}
\label{sec:interfaceProblemTree}
%%%%%%%%% %%%%%%%%%%%%%%%%%%%%%%%%%%

Overall, the problem tree is given as follows:

\begin{itemize}
\item Policy core problems:
	\begin{enumerate}
	
	\item Sheep - 
	The amount of sheep on the grid.
	
	\item Wolf - 
	The amount of wolves on the grid.
	
	\item Fully grown grass - 
	The amount of fully grown grass on the grid.
	
	\end{enumerate}
	
\item Secondary problems:
	\begin{enumerate}

	\item Net sheep population change - 
	This is the difference between initial and final amount of sheep.
	
	\item Net wolf population change - 
	This is the difference between initial and final amount of wolves.
	
	\item Net grown grass patch change - 
	This is the difference between initial and final amount of grown grass patches.
	
	\end{enumerate}
	
\end{itemize}

Note that the secondary issues and the policy core issues are the same. The main reason behind this choice is the simplicity of the model. The model is so simple that the it is difficult to find any different policy core issues.

%%%%%%%%%%%%%%%%%%%%%%%%%%%%%%%%%%%%%%%%%%%%%%%%%%%%%%%%%%%%%%%%%%
\section{The policy instruments}
\label{sec:interfaceInstruments}
%%%%%%%%% %%%%%%%%%%%%%%%%%%%%%%%%%%

The policy instruments within the policy tree are implemented using incremental increases and decreases in the following exogenous parameters.

\begin{enumerate}
\item Change sheep reproduction [CR-0.01/+0.01]
\item Change wolf reproduction [WR -0.01/+0.01]
\item Change grass regrowth [GR-2/+2]
\end{enumerate}



%%%%%%%%%%%%%%%%%%%%%%%%%%%%%%%%%%%%%%%%%%%%%%%%%%%%%%%%%%%%%%%%%%
\section{The steps for model integration}
\label{sec:steps}
%%%%%%%%% %%%%%%%%%%%%%%%%%%%%%%%%%%

This section presents the steps that are needed to connect a policy context model, in this case the predation model, to the policy process model.

\begin{enumerate}
\item Before any coding, define what the belief tree and the policy instruments will be for the predation model.
\item Copy the policy emergence model files into the same folder.
\item In \texttt{runbatch.py}, replace the policy context items by the predation model.
\item In \texttt{runbatch.py}, make sure to initialise the predation model appropriately.
\item Change the \texttt{input goalProfiles} files to have the appropriate belief tree structure of the predation model.
\item In \texttt{model module interface.py}, construct the belief tree and the policy instrument array.
\item Make sure that the step function in the \texttt{model predation.py} returns the KPIs that will fit in the belief system in the order DC, PC and S. If no DC is considered, then include one value of 0 at least. All KPIs need to be normalised.
\item Modify the step function of the \texttt{model predation.py} to include a policy implemented.
\item Introduce the changes that a policy implemented would have on the model in \texttt{model predation.py}.
\end{enumerate}


%%%%%%%%%%%%%%%%%%%%%%%%%%%%%%%%%%%%%%%%%%%%%%%%%%%%%%%%%%%%%%%%%%
\section{The steps for model simulation}
\label{sec:steps}
%%%%%%%%% %%%%%%%%%%%%%%%%%%%%%%%%%%

This section presents the steps that are needed to connect a policy context model, in this case the predation model, to the policy process model.

\begin{enumerate}
\item For the policy process:
	\begin{enumerate}
	\item Define a set of hypotheses to be tested
	\item Define scenarios that will be needed to assess the hypotheses
	\item Choose the agent distribution based on the scenarios constructed
	\item Set the preferred states for the active agents and the electorate along with the causal beliefs to be used. This should all be based on the scenarios that have been constructed.
	\end{enumerate}

\item For the predation model:
	\begin{enumerate}
	\item Define the initial values for the main parameters
	\item Define the parameters that will be recorded
	\end{enumerate}
\item Save the right data from the model.
\end{enumerate}

%%%%%%%%%%%%%%%%%%%%%%%%%%%%%%%%%%%%%%%%%%%%%%%%%%%%%%%%%%%%%%%%%%
\section{Model hypotheses}
\label{sec:hypotheses}
%%%%%%%%% %%%%%%%%%%%%%%%%%%%%%%%%%%

Because of the complexity of the model, there are a lot of hypotheses that can be tested with this ACF policy emergence model. However, we will try not to demonstrate the same hypotheses that were demonstrated for the SM. This would be redundant. This includes all of the hypotheses that establish a causal relation between policy change and environment change. The focus will instead be placed on the impact that the new elements, for each of the model variations, has on policy change and on the dynamics of the model. This includes the impact of policy learning and coalitions on policy change.

%%%%%%%%%%%%%%%%%%%%%%%%%%%%%%%%%%%%%%%%%%%%%%%%%%%%%%%%%%%%%%%%%%
\subsection{The +PL model}

The main research question that we want to answer with this model is:

\emph{What impact does policy learning have on policy change within the ACF+PL model?}

Several hypotheses are considered to answer this question:

\begin{itemize}
\item H1: Policy learning leads to policy change.
\item H2: The resources distribution will affect the speed and the direction of policy learning.
\item H4: The balance of power will affect the speed and the direction of policy learning.
\end{itemize}

%%%%%%%%%%%%%%%%%%%%%%%%%%%%%%%%%%%%%%%%%%%%%%%%%%%%%%%%%%%%%%%%%%
\subsection{The +Co model}

The main research question that we want to answer with this model is:

\emph{What impact do coalitions have on policy change within the ACF+Co model?}

Several hypotheses are considered to answer this question:

\begin{itemize}
\item H1: 
\end{itemize}


%%%%%%%%%%%%%%%%%%%%%%%%%%%%%%%%%%%%%%%%%%%%%%%%%%%%%%%%%%%%%%%%%%
\section{Model scenarios}
\label{sec:steps}
%%%%%%%%% %%%%%%%%%%%%%%%%%%%%%%%%%%

We differentiate the scenarios that will be run for the +PL and the +Co models within this section. For each scenario, we have to consider the belief system of the agents, the agent distribution and the resource distribution. Though in the formalisation of the model we do not make a mention of affiliations (except when it comes to the electorate influence), we will still use affiliations here to simplify the initialisation of the agents for the policy emergence model.

%%%%%%%%%%%%%%%%%%%%%%%%%%%%%%%%%%%%%%%%%%%%%%%%%%%%%%%%%%%%%%%%%%
\subsection{The +PL model}

Six scenario are considered. All but one focus on a change in the preferred states of the agents or their causal beliefs. For each of the scenario, the preferred states of the agents are shown in \autoref{tab:preferredStates} and their causal relations are provided in \autoref{tab:causalBeliefs}.


\begin{itemize}
\item Scenario 0 - Benchmark

The benchmark scenario is to be used as a benchmark. It is a simulation of the predation model with the policy emergence model. The preferred states for the agents is provided in \autoref{tab:preferredStates}. The causal beliefs used as given in \autoref{tab:causalBeliefs}.

\item Scenario 1 - Demonstrating policy learning but not policy change

The aim of this scenario is to demonstrate that policy learning does not always lead to policy change by having a resource rich set of agents not be the dominant set of agents for the agenda selection and the policy implementation.

%Scenario 1 looks at what would change if the policy core issue preferred states of the policy makers were different. The new selection of preferred states is given in \autoref{tab:preferredStates}.

\item Scenario 2 - Change in resource distribution

Scenario 2 looks at what would change if the secondary issue preferred states of the policy makers were different. The new selection of preferred states is given in \autoref{tab:preferredStates}.

\item Scenario 3 - Change in the causal beliefs

For Scenario 3, we compare a difference in the understanding of how the system works and its impact on policy change. For this we create a different causal beliefs structure that is presented in \autoref{tab:causalBeliefs}. This is to be compared with results from the benchmark in scenario 0.

\item Scenario 4 \& 5 - Electorate influence on the policy core and secondary issue preferred states

For scenario 3, we also check what is the impact of different electorate influence weight values. The aim is to visualise the impact of the influence electorate on the overall system for this. We have the same scenario as for scenario 2 but this time the electorate is much more driven and has a quick impact on the policy makers (quadrupling to 0.20 their influence on policy makers). The weights considered are 0.20, 0.02 and 0.50.

\end{itemize}


% max sheep: [0, 500]
% max wolves: [0, 500]
% mass grass patches: [2500]
% Net sheep population change: [-100; 100]
% Net wolf population change: [-100; 100]
% Net grown grass patch change: [-500; 500]

\begin{table}[h!]
\begin{center}
\begin{tabular}{ |c|c|c|c|c|c|c| } 
\hline

			& PC1	& PC2	& PC3	& S1		& S2		& S3  	\\ 
			& Sheep	& Wolves	& Grass	& Sheep	& Wolves	& Grass 	\\
			&		&		&		& growth	& growth	& growth	\\ \hline \hline
			
	 		& \multicolumn{6}{|c|}{Affiliation 0}						\\ \hline 
Agents		& 300	& 300	& 2000	& 100	& 50		& 200	\\ \hline
			& 0.60	& 0.60	& 0.80	& 1.00	& 0.75	& 0.70	\\ \hline
	 		& \multicolumn{6}{|c|}{Affiliation 1}						\\ \hline 
Agents		& 300	& 300	& 2000	& 100	& 50		& 200	\\ \hline
			& 0.60	& 0.60	& 0.80	& 1.00	& 0.75	& 0.70	\\ \hline
			
\multicolumn{7}{|c|}{Scenario 1}										\\ \hline
Policy makers	& 150	& 0		& 2000	& 100	& 50		& 200	\\ \hline
			& 0.30	& 0.00	& 0.80	& 1.00	& 0.75	& 0.70	\\ \hline
			


\end{tabular}
\end{center}
\caption{Preferred states for the policy makers on a the interval [0,1] for scenarios 0 and 1.}
\label{tab:preferredStates}
\end{table}

\begin{table}[h!]
\begin{center}
\begin{tabular}{ |c|c|c|c| |c|c|c|c|}
 \hline
\multicolumn{4}{|c||}{Scenario 0/1/2/4/5}	& \multicolumn{4}{|c|}{Scenario 3}		\\ \hline
	& PC1	& PC2	& PC3		& 		& PC1	& PC2	& PC3	\\ \hline
-S1 	& 1.00	& 0.75	&-0.75		& -S1 	&-0.50	&-0.10	& 0.25	\\ \hline
-S2 	&-0.75	& 1.00	& 0.25 		& -S2 	& 0.05	&-0.50	&-0.25 	\\ \hline
-S3 	& 0.50	& 0.75	& 1.00		& -S3 	&-0.25	& 0.00	&-0.50	\\ 
 \hline
\end{tabular}
\end{center}
\caption{Causal beliefs for the policy makers. These causal relations can be read as: an increase of 1 in S2 will lead to a decrease of 0.75 in PC1. They are all given on the interval [-1,1].}
\label{tab:causalBeliefs}
\end{table}

\begin{table}[h!]
\begin{center}
\begin{tabular}{ |c|c|c|c|c|c|c|c|c|c|c|c|c| } 
\hline

Scenarios		& \multicolumn{2}{|c|}{0}	
					& \multicolumn{2}{|c|}{1}	
							& \multicolumn{2}{|c|}{2}	
									& \multicolumn{2}{|c|}{3}	
											& \multicolumn{2}{|c|}{4}	
													& \multicolumn{2}{|c|}{5}	 
															\\ \hline \hline
Affiliations		& 0	& 1	&  0	& 1	&  0	& 1	&  0	& 1 	&  0	& 1	&  0	& 1	\\ \hline
Policy makers 	& 2	& 1	& 2	& 1	& 	&	&	&	&	&	&	& 	\\ \hline
Policy entrepreneurs
			& 4	& 4	& 4	& 4	& 	&	&	&	&	&	&	& 	\\ \hline
\end{tabular}
\end{center}
\caption{Agent distribution for each of the scenarios.}
\label{tab:agentDistribution}
\end{table}



%%%%%%%%%%%%%%%%%%%%%%%%%%%%%%%%%%%%%%%%%%%%%%%%%%%%%%%%%%%%%%%%%%
\subsection{The +Co model}


%%%%%%%%%%%%%%%%%%%%%%%%%%%%%%%%%%%%%%%%%%%%%%%%%%%%%%%%%%%%%%%%%%
\section{Initialisation of the predation model}
\label{sec:predation_initialisation}
%%%%%%%%%%%%%%%%%%%%%%%%%%%%%%%%%%%

The parameters that need to be initialised for the predation model are given by:

\begin{itemize}
\item Grid height: 50
\item Grid width: 50
\item Initial amount of grass: about 50\% of the grid
\item Initial number of sheep: 250
\item Sheep reproduce rate: 4\%
\item Sheep gain from food: 6
\item Initial number of wolves: 25
\item Wolf reproduce rate: 5\%
\item Wolf gain from food: 35
\item Grass regrowth time: 30
\end{itemize}

Note that the initial parameter as chosen such that if only the predation model is run, it has a stable configuration. Furthermore the onus is placed on the simulation of the policy process, therefore no scenarios are placed on the predation model side of the simulation.


%%%%%%%%%%%%%%%%%%%%%%%%%%%%%%%%%%%%%%%%%%%%%%%%%%%%%%%%%%%%%%%%%%
\section{Initialisation of the policy emergence model}
\label{sec:}
%%%%%%%%%%%%%%%%%%%%%%%%%%%%%%%%%%%

Conflict level thresholds ...

Resources spent



%%%%%%%%%%%%%%%%%%%%%%%%%%%%%%%%%%%%%%%%%%%%%%%%%%%%%%%%%%%%%%%%%%
\bibliographystyle{apalike} 
\bibliography{references}
%%%%%%%%%%%%%%%%%%%%%%%%%%%%%%%%%%%%%%%%%%%%%


\end{document}
