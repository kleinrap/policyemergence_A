\documentclass[12pt]{article}
\usepackage[utf8]{inputenc}
\usepackage{longtable}
\usepackage{color}
\usepackage{multirow}
\usepackage{hyperref}
\usepackage{setspace}
%\doublespacing

\renewcommand*{\sectionautorefname}{Section} 

\title{Electricity model\\ Hybridisation\_A} 
\author{Raphael Klein}

\usepackage{natbib}
\usepackage{graphicx}
\usepackage[labelfont=bf]{caption} 	% Make captions bold (Figure & Table)
\usepackage{subfig}	
\usepackage{amsmath}
\usepackage{hyperref}
%\usepackage[section]{placeins}

\providecommand{\keywords}[1]{\textbf{Keywords:} #1}

\begin{document}

\maketitle

%\textcolor{red.green.blue.cyan.yellow.magenta.}{}

%\newpage

% structure of the report
% Introduction
% Modelling
%	Electricity model
%	Policy emergence model (SM)
%	Hybrid model
% Implementation
%	Electricity model
%	Policy emergence model (SM)
%	Hybrid model
% Code documentation
%	Electricity model
% 	Policy emergence model (SM)
%	Hybrid model
% Verification
%	Electricity model
% 	Policy emergence model (SM)
%	Hybrid model
% Inputs
%	Electricity model
%	Policy emergence model (SM)
%	Hybrid model
% Experimentation - Scenario design
% Model initialisation
%	Electricity model
% 	Policy emergence model (SM)
%	Hybrid model
% Model results 
% Conclusions



This reports the hybridisation of the electricity model with the simplest implementation of the policy model. It outlines how the models were hybridised and goes through the initialisation of the models and the experiments that will be run using these models.

\tableofcontents

%%%%%%%%%%%%%%%%%%%%%%%%%%%%%%%%%%%%%%%%%%%%%%%%%%%%%%%%%%%%%%%%%%
\section{The problem tree}
\label{sec:interfaceProblemTree}
%%%%%%%%% %%%%%%%%%%%%%%%%%%%%%%%%%%


To couple the models, the agents in the policy process need to be provided with a problem tree. This problem tree is specific to the electricity market model as it is informed by the key performance indicators from that model. This was highlighted in the previous section. Beyond this, the problems selected are also informed from previous work done by \cite{markard2016socio}. They identified a number of problems (they call these beliefs in their publication) that are specific to the Swiss context. These are however limited to the deep core and policy core levels. Secondary problems are not included or researched and are therefore taken from the model only.

The difficulty in the creation of the problem tree is to associate the right indicators to the right problems. The first step is to not consider the deep core problems. These are considered to be normative problems. They are beyond the boundaries of the model, out of the scope. They are not a crucial aspect of the process as it is focused on policy core problems so it does not make a big difference if deep core problems are considered or not. The next step is to consider the secondary problems. These can be found directly within the model. They are indicators that are made into secondary problems. Not all indicators are considered, only a few are selected. These are considered to be the important ones for the agents in the policy process. Finally, there is the selection of the policy core problems. These are, in general, aggregates of the secondary problems. They are calculated as a function of the main model indicators.

For the policy core problems, there is an additional aspect that needs to be considered. In work performed by \citeauthor{markard2016socio}, policy core problems within the Swiss electricity market subsystem were identified. These are: seriousness of the problem, role of the state, environment, economy and society. Several of these cannot be obtained from the model as they are outside of the boundaries of the model. However, the environment and economy can be considered. They are therefore selected as the policy core problems. \cite{markard2016socio} also identified four secondary problems. They are however not suitable for the model as they are more questions than problems. Furthermore, four secondary problems is not sufficient. It is for this reason that the secondary problems are only selected from the model. Ultimately, the policy core problems are calculated using linear equations that include a number of the indicators used for the secondary problems.

Overall, the problem tree is given as follows:

\begin{itemize}
\item Policy core problems:
	\begin{itemize}
	\item Economy
	\item Environment
	\end{itemize}
\item Secondary problems:
	\begin{itemize}
	\item Renewable energy production
	\item Electricity prices
	\item Renewable energy investments
	\item Domestic level emissions
	\item Imported emissions
	\end{itemize}
\end{itemize}

The economy takes into account elements related to profits of firms along with the security of supply of the country. The environment takes into account aspects such as the emissions, the amount of renewable energy and the amount of imported emissions.




\subsection{Implementation of the KPIs}

The KPIs are calculated within the electricity model but they are only used for the hybrid model. There are five secondary indicators that are calculated and two policy core indicators. The secondary indicators are calculated directly from the data that is obtained from the model while the policy core indicators are calculated based on the secondary indicators. Each of the indicators are also normalised as they are to be used within the belief system of the actors within the policy process model. All indicators are calculated for data that is obtained for the year prior to the policy making round. The data considered does not include all of the years between negotiating rounds.

The secondary indicators are the following: renewable energy production (S1), electricity prices (S2), renewable energy investment level (S3), domestic level emissions (S4) and imported emissions (S5).

The renewable energy production (S1) indicator is calculated as follows. The total supply of electricity is given by: (note we only consider domestic production and no imports or exports)

\begin{equation}
S_{total} = S_{solar} + S_{CCGT} + S_{wind} + S_{nuclear} + S_{hydro} + S_{hydrop} + S_{waste} + S_{ROR}
\end{equation}

The renewable supply is given by:

\begin{equation}
S_{RES} = S_{solar} + S_{wind} S_{hydro} + S_{hydrop} S_{ROR}
\end{equation}

The indicator is then normalised using:
\begin{equation}
S1 = S_{RES}  / S_{total}
\end{equation}

The electricity prices (S2) indicator is calculated based on the average electric price of the previous year. It is then normalised using an expected maximum electricity price ($P_{elec, max}$). The normalisation equation is given by:

\begin{equation}
S2 = \frac{P_{elec}}{P_{elec, max}}
\end{equation}

$P_{elec, max}$ is selected to be equal to 150 but can be tuned depending on the outcome of simulation such that the values of S2 are always between 0 and 1.

The renewable energy investment level (S3) indicator is calculated using the investment performed by all the investors in solar, wind and CCGT assets.

\begin{equation}
I_{total} = I_{wind} + I_{solar} + I_{CCGT}
\end{equation}

\begin{equation}
I_{RES} = I_{wind} + I_{solar}
\end{equation}

The indicator is normalised using:

\begin{equation}
S3 = I_{RES} / I_{total}
\end{equation}

The domestic level emissions (S4) is calculated based on the CCGT emissions. The normalisation of this indicator is once again done using an assumed maximum for the emissions which is given as five times the emissions for year 1 of the simulation. This is an arbitrary value that can be tuned to make sure that the indicator is always between 0 and 1.

\begin{equation}
S4 = S_{CCGT} / S_{CCGT, max}
\end{equation}

The imported emissions (S5) indicator is calculated based on the imports and the policy mixes of the countries from which Switzerland imports. The policy mixes are scenarios that are obtained from technical report and goals for the different countries. For each country, the percentage of coal and gas production is considered to calculated the imported emissions.

\begin{subequations}
\begin{align}
        E_{FR} & = (S_{FR, NTC} + S_{LTC}) \cdot (M_{FR, CCGT} \cdot E_{CCGT} + M_{FR, coal} \cdot E_{coal}) \\
         E_{DE} & = Ss_{DE, NTC} \cdot (M_{DE, CCGT} \cdot E_{CCGT} + M_{DE, coal} \cdot E_{coal}) \\
         E_{IT} & = S_{IT, NTC} \cdot (M_{IT, CCGT} \cdot E_{CCGT} + M_{IT, coal} \cdot E_{coal})
         \end{align}
\end{subequations}

where $E$ are the emissions per type of technology and where $M$ is the share of the mix for a specific technology in the country.

To normalise this indicator, we once again select a maximum amount of emissions. This is calculated as being 5\% higher than the initial imported emissions in year 1 of all these countries. Considering the emissions should decrease in the scenarios, this means that the indicator should remain within the $[0, 1]$ interval.

\begin{equation}
S5 = \frac{ E_{FR} + E_{DE} + E_{IT} }{ E_{total}}
\end{equation}

where $E_{total} = E_{FR, init} + E_{DE, init} + E_{IT, init}$


The policy core issues are given as the economy (PC1) and the environment (PC2). They are calculated using weighted averages of a number of secondary indicators. The equations used can be tuned but the ones implemented are given below:

\begin{equation}
PC1 = \frac{3}{4} \cdot S2 + \frac{1}{4} \cdot S3
\end{equation}

\begin{equation}
PC2 = \frac{1}{4} \cdot S1 + \frac{1}{4} \cdot S3 + \frac{1}{4} \cdot S4 + \frac{1}{4} \cdot S5
\end{equation}








%%%%%%%%%%%%%%%%%%%%%%%%%%%%%%%%%%%%%%%%%%%%%%%%%%%%%%%%%%%%%%%%%%
\section{The policy instruments}
\label{sec:interfaceInstruments}
%%%%%%%%% %%%%%%%%%%%%%%%%%%%%%%%%%%

The policy instruments within the policy tree are implemented using incremental increases and decreases in the following exogenous parameters.

\begin{enumerate}
\item Solar subsidies [+/- 0.04]
\item Wind turbine permit times [+/- 0.04]
\item Agent's hurdle rate  [+/- 0.02]
\item Carbon tax on domestic fossil fuel [+/- 10]
\item Carbon tax on fossil fuel imports [+/- 10]
\end{enumerate}


%%%%%%%%%%%%%%%%%%%%%%%%%%%%%%%%%%%%%%%%%%%%%%%%%%%%%%%%%%%%%%%%%%
\section{The steps for model integration}
\label{sec:steps}
%%%%%%%%% %%%%%%%%%%%%%%%%%%%%%%%%%%

This section presents the steps that are needed to connect a policy context model, in this case the predation model, to the policy process model.

\begin{enumerate}
\item Before any coding, define what the belief tree and the policy instruments will be for the predation model.
\item Copy the policy emergence model files into the same folder.
\item In \texttt{runbatch.py}, replace the policy context items by the predation model.
\item In \texttt{runbatch.py}, make sure to initialise the predation model appropriately.
\item Change the \texttt{input goalProfiles} files to have the appropriate belief tree structure of the predation model.
\item In \texttt{model module interface.py}, construct the belief tree and the policy instrument array.
\item Make sure that the step function in the \texttt{model predation.py} returns the KPIs that will fit in the belief system in the order DC, PC and S. If no DC is considered, then include one value of 0 at least. All KPIs need to be normalised.
\item Modify the step function of the \texttt{model predation.py} to include a policy implemented.
\item Introduce the changes that a policy implemented would have on the model in \texttt{model predation.py}.
\end{enumerate}

%%%%%%%%%%%%
\subsection{Code documentation}

The following is the documentation for the hybrid model. This includes the script files that are needed to connect the policy context model (electricity model in the present case) with the policy emergence model.


\paragraph{run\_batch.py}

This script is used to simulate the entire hybrid model for different scenario. To this effect, it contains the inputs for all models. This includes the inputs for the hybrid model itself (number of steps, duration of steps, number of repetitions, number of scenarios, ...), the inputs for the policy context, and the inputs for the policy emergence model (actor distribution, actor belief profiles, ...)

This is then followed by the for loop that simulate the hybrid model. This includes the simulation of a warm up round. Then the policy context is simulated n times for every policy process step simulation. Each feeds the other through indicators and policy selection. The results are all extracted using the data collector from each of the model. The files are saved within .csv files.

\paragraph{model\_module\_interface.py}

This script is used to connect the policy context to the policy emergence model. This part of the model will change every time a new policy context is considered. For this, two functions are considered:

\begin{itemize}
\item \texttt{belief\_tree\_input()}

This function is used to define the agent issue tree. It includes the specification of the deep core, policy core and secondary issues.

\item \texttt{policy\_instrument\_input()}

This function is used to define the policy instruments that the agents can select.

\end{itemize}


%%%%%%%%%%%%%%%%%%%%%%%%%%%%%%%%%%%%%%%%%%%%%%%%%%%%%%%%%%%%%%%%%%
\section{The steps for model simulation}
\label{sec:}
%%%%%%%%%%%%%%%%%%%%%%%%%%%%%%%%%%%

This section presents the steps that are needed to connect a policy context model, in this case the electricity model, to the policy process model.

\begin{enumerate}
\item For the policy process:
	\begin{enumerate}
	\item Define a set of hypotheses to be tested
	\item Define scenarios that will be needed to assess the hypotheses
	\item Choose the agent distribution based on the scenarios constructed
	\item Set the preferred states for the active agents and the electorate along with the causal beliefs to be used. This should all be based on the scenarios that have been constructed.
	\end{enumerate}

\item For the predation model:
	\begin{enumerate}
	\item Define the initial values for the main parameters
	\item Define the parameters that will be recorded
	\end{enumerate}
\item Save the right data from the model.
\end{enumerate}

%%%%%%%%%%%%%%%%%%%%%%%%%%%%%%%%%%%%%%%%%%%%%%%%%%%%%%%%%%%%%%%%%%
\section{Model verification}
\label{sec:}
%%%%%%%%%%%%%%%%%%%%%%%%%%%%%%%%%%%

The following is the verification for the policy process model.

\begin{longtable}{|l|p{6cm}|c|c|}
\hline
\textbf{Functions} & \textbf{Issues} & \textbf{Verified}  \\
\hline \hline
\endfirsthead
\multicolumn{3}{c}%
{\tablename\ \thetable\ -- \textit{Continued from previous page}} \\
\hline
\textbf{Functions} & \textbf{Issues} & \textbf{Verified}  \\
\hline \hline
\endhead
\hline \multicolumn{3}{c}{\textit{Continued on next page}} \\
\endfoot
\hline
\endlastfoot
\multicolumn{2}{|c|}{{\bfseries model\_SM.py}}	& Yes	\\ \hline
get\_agents\_attributes 		& None		& Yes	\\ \hline
get\_electorate\_attributes		& None		& Yes	\\ \hline
get\_problem\_policy\_chosen	& None		& Yes	\\ \hline
step						& None		& Yes	\\ \hline
module\_interface\_input		& None		& Yes	\\ \hline
agenda\_setting			& None		& Yes	\\ \hline
policy\_formulation			& None 		& Yes	\\ \hline
preference\_update			& None		& Yes	\\ \hline
preference\_update\_DC		& None		& Yes	\\ \hline
preference\_update\_PC		& None		& Yes	\\ \hline
preference\_update\_S		& None		& Yes	\\ \hline
electorate\_influence			& None		& Yes	\\ \hline

\multicolumn{2}{|l|}{{\bfseries model\_SM\_agents.py}}
									& Yes	\\ \hline
selection\_PC				& None		& Yes	\\ \hline
selection\_S				& None		& Yes	\\ \hline
selection\_PI				& Was missing a check for the agenda - has been added
									& Yes	\\ \hline
electorate\_influence			& None		& Yes	\\ \hline

\multicolumn{2}{|l|}{{\bfseries model\_SM\_agents\_initialisation.py}}
									& Yes	\\ \hline
issuetree\_creation			& None		& Yes	\\ \hline
policytree\_creation			& None		& Yes	\\ \hline
init\_active\_agents			& None		& Yes	\\ \hline
init\_electorate\_agents		& None		& Yes	\\ \hline
init\_electorate\_agents		& None		& Yes	\\ \hline
init\_truth\_agent			& None		& Yes	\\ \hline
						& None		& Yes	\\ \hline
\multicolumn{2}{|l|}{{\bfseries model\_SM\_policyImpact.py}}
									& Yes	\\ \hline
model\_simulation			& None		& Yes	\\ \hline
policy\_impact\_evaluation	& Some code simplification were performed
									& Yes	\\ \hline
%	&			&	\\ \hline
	
\end{longtable}


%%%%%%%%%%%%%%%%%%%%%%%%%%%%%%%%%%%%%%%%%%%%%%%%%%%%%%%%%%%%%%%%%%
\section{Model hypotheses/Research questions}
\label{sec:hypotheses}
%%%%%%%%% %%%%%%%%%%%%%%%%%%%%%%%%%%

An hypothesis for this model is provided below:

\begin{itemize}
\item H1: Without additional resources or a shift in the views of the pro-economy coalition, the energy transition goals will be difficult, if not impossible, to achieve.
\end{itemize}

This hypothesis can be used to create a research questions and the subsequent scenarios. Note that the electorate is not considered within this approach as it was in the simplest implementation. The following question is formulated:

Can the goals of the Swiss energy transition in the electricity market be met with the current power dynamics between the subsystem coalitions?

%%%%%%%%%%%%%%%%%%%%%%%%%%%%%%%%%%%%%%%%%%%%%%%%%%%%%%%%%%%%%%%%%%
\section{Model scenarios}
\label{sec:steps}
%%%%%%%%% %%%%%%%%%%%%%%%%%%%%%%%%%%

Scenarios are built to answer the research question. Similarly to what was done for the simplest implementation, the scenarios for the electricity market model need to take into account scenarios for the electricity model and for the policy process model as both models are complex and are needed to answer the research question.

Before the scenarios are decided, it is important to decide which ACF implementation model should be used. Four versions were presented at different levels of complexity included two add-ons. Unlike for the predation model, the aim is not to test the potential of the policy process but instead to answer a question specific to the electricity model. It is therefore needed to decide which version of the policy model should be used to answer that question without using each of the version unnecessarily. Considering the electricity market model and the Swiss conditions, it is possible to assume that agents have perfect information and full knowledge with the policy subsystem considered here. This is because electricity is not that much of a polarising issue in Switzerland as other issues might be. Furthermore, a lot of the data on the electricity sector is obtained from the government and available online freely. It is therefore harder to distort such realities. Of course this assumption is a little optimistic but it should not affect the outcomes of the model materially. This means that add-ons for partial knowledge and partial information do not need to be considered. Coalitions on the other hands are needed as was shown in \cite{markard2016socio}. Two prominent coalitions exists within the subsystem: a pro-environment coalition and a pro-economy coalition. Therefore the coalition model will be used, which includes the policy learning aspects of the model. Furthermore, the initialisation of the model will be made along two affiliations that do not necessarily represent the parties but that represent the coalition beliefs.

For the electricity market model, the scenarios considered are kept simple. They reflect the drive for electrification that were already looked at for the simplest implementation model. This include electricity demand growth of 0\%, 1.5\% and 3\%.

For the policy process, a different approach than for the simplest implementation is used. The electorate is not considered at all within this ACF implementation. Instead, the focus is put on the resources that the agents have. According to \cite{markard2016socio}, the resources of the pro-economy coalitions are larger, and they also benefit from a larger coalition in numbers. The scenarios devised here should try to invert this trend to show how this might impact the electricity market and allow me to answer the research question.

One part is the policy process: benchmark with normal current conditions AND switch beliefs of the economy agents to be more environmentally friendly midway AND compare different resources distribution where the environmental groups have more resources

Also ... consider the power distribution ... how does the number of agents affect the outcomes of the model ... especially with the entrepreneurs and the policy makers ... maybe find some inspiration from the agents that are considered in the \cite{markard2016socio} paper. First use the same number of agents as they identify (policy makers and entrepreneurs) and then see how this might need to be adjusted/limited/augmented for the model depending on the parliament forces.

The scenarios are therefore given as follows:

\begin{itemize}
\item Scenario 0 - Benchmark - run without changes in the beliefs or resources of the agents
\item Scenario 1 - Run with changes in the preferred states of the agents midway.
\item Scenario 2 - Run with changes in the resources of the agents from the start.
\end{itemize}


\begin{table}
\begin{center}
\begin{tabular}{|l|c|c|c|c|c|c|c|} 
\hline
			& PC1 	& PC2	& S1		& S2			& S3		& S4			& S5			\\ 
			&Eco	& Env	& RES	& Price		& REI	& Dom. em.	& Imp. em.	\\
			& [-]		& [-]		& [\%]	& [CHF/MWh]	& [\%]	& [tons]		& [tons]		\\ \hline
\multicolumn{8}{|l|}{2016 preferred states}													\\ \hline
Coal. Eco.		& 		& 		& 60		& 50 			& 70		& 4 000 000	& 60 000 		\\ \hline
			& 		& 		& 0.60	& 0.75 		& 0.70	& 0.80		& 0.33 		\\ \hline
Coal. Env. 	& 		& 		& 100	& 75			& 100	& 0			& 5 000		\\ \hline
			& 		& 		& 1.00	& 0.63		& 1.00	& 1.00		& 0.94 		\\ \hline
\multicolumn{8}{|l|}{2021 conditions}														\\ \hline
Coal. Eco.		& 		& 		& 95		& 55 			& 95		& 500 000		& 7 500 		\\ \hline
			& 		& 		& 0.95	& 0.73		& 0.95 	& 0.98		& 0.99 		\\ \hline
Coal. Env. 	& 		& 		& 100	& 75			& 100	& 0			& 5 000		\\ \hline
			& 		& 		& 1.00	& 0.63		& 1.00	& 1.00		& 0.94 		\\ \hline
\end{tabular}
\end{center}
\caption{Preferred states for the policy makers and the electorate on a the interval [0,1] for all scenarios..}
\label{tab:si_elec_beliefs}
\end{table}

\begin{table}
\begin{center}
\begin{tabular}{ |c|c|c|c| |c|c|c|c|}
 \hline
\multicolumn{3}{|c|}{Coalition eco.}
					& \multicolumn{3}{|c|}{Coalition env.}	\\ \hline \hline
	& PC1	& PC2	&		& PC1	& PC2	\\ \hline
-S1 	& 0.00	& 0.25	& -S1 	& 0.00	& 0.25	\\ \hline
-S2 	& 0.75	& 0.00	& -S2 	& 0.75	& 0.00	\\ \hline
-S3 	& 0.25	& 0.25	& -S3 	& 0.25	& 0.25	\\ \hline
-S4 	& 0.00	& 0.25	& -S4 	& 0.00	& 0.25	\\ \hline
-S5 	& 0.00	& 0.25	& -S5 	& 0.00	& 0.25	\\
 \hline
\end{tabular}
\end{center}
\caption{Causal beliefs for the policy makers for all scenarios. They are all given on the interval [-1,1]. \textcolor{red}{Change this, not yet adjusted.}}
\label{tab:si_elec_causalBeliefs}
\end{table}


The different issues are normalised using the following criteria which are dependent on the ranges that the parameters can reach in the simulation: 

\begin{itemize}
\item S1: renewable energy production on range [0, 1].
\item S2: electricity prices on range [200, 0]. % init 40
\item S3: renewable energy investment level on range [0, 1].
\item S4: domestic level emissions on range [0, $\sim$20m].
\item S5: imported emissions on range [0, $\sim$9m]. % init max 60k
\item PC1: economy [0, 1]. % init 0.57 PC1 = 0.75 * S2 + 0.25 * S3
\item PC2: environment [0, 1]. % init 0.392/0.393 PC2 = 0.25 * S1 + 0.25 * S3 + 0.25 * S4 + 0.25 * S5
\end{itemize}

To inform the selection of the agent distribution between the two coalition-based affiliations to initialise the model, the paper from \cite{markard2016socio} is used. In this paper, a distribution of actor is provided based on surveys and interviews. The distribution of actors along the coalitions is given as follows:

\begin{itemize}
\item Pro-economy coalition:
	\begin{itemize}
	\item Policy makers (4): BDP, CVP, FDP, SVP
	\item Policy entrepreneurs (15): Economiesuisse, EV, SGV, Swissmem, Alpiq, Axpo, BKW, EWZ, Swissgrid, Swisspower, Energieforum, IGEB, Swisselectric, VSE, ETH-Rat
RKGK
	\end{itemize}
\item Pro-environment coalition:
	\begin{itemize}
	\item Policy makers (3): GLP, GPS, SP
	\item Policy entrepreneurs (8): SGB, Swisscleantech, Pro Natura, VCS, WWF, AEE Suisse, SES, AkadWiss
	\end{itemize}
\end{itemize}

The policy makers do not represent the proportions that the parties have in parliament, however it does represent well the fact that overall, the pro-economy coalition has more policy makers than the pro-environment coalition, leaving them with more power in the agenda setting process and the policy selection within the context of the model. For the policy entrepreneurs, the ratio of entrepreneurs for the coalition can be used as is as it would accurately represent the different in agents.

For the model, the agent distribution will be the same as in the paper. For the pro-economy coalition, this will be four policy makers and fifteen policy entrepreneurs while the pro-environment coalition will have three policy makers and eight policy entrepreneurs.



A PC needs to be chosen for the +Co model. The other +Co parameters also need to be considered.


%%%%%%%%%%%%%%%%%%%%%%%%%%%%%%%%%%% end of the section

%%%%%%%%%%%%%%%%%%%%%%%%%%%%%%%%%%%%%%%%%%%%%%%%%%%%%%%%%%%%%%%%%%
\section{Initialisation of the model}
\label{sec:elec_initialisation}
%%%%%%%%%%%%%%%%%%%%%%%%%%%%%%%%%%%

\textcolor{red}{This section needs to be adjusted for the ACF+PL, +Co, +PK, +PI.}


%%%%%%%%%%%%
\subsection{The policy process model}

The affiliations, the actor distribution and their preferred state need to be initialised for the policy process model. These elements are not part of the scenarios and are therefore constant across all simulations.

\paragraph{The affiliations}

There are two affiliations that need to be considered for the electricity sector in the policy process model. These follow the findings of \cite{markard2016socio}. One of the affiliation is focused on the economy (affiliation 1) while the other on the environment (affiliation 2). Their differences in beliefs outlined in \cite{markard2016socio} will be reflected in their preferred states. Note that no surveys were performed for this study as this is an initial study and it is considered that the study made in \cite{markard2016socio} is still sufficiently recent to apply to the model at hand here.


\paragraph{The actor distribution}

It is assumed in this approach that the actor distribution will not change over time. Only the beliefs of the electorate and the actors change. This is an assumption that could have an impact on the results obtained. This is translated in 3 policy makers and 4 policy entrepreneurs (Affiliation 1: 2 policy maker and 2 policy entrepreneurs; affiliation 2: 1 policy makers and 2 policy entrepreneurs).

Because of computational efficiency issues, not all actors that were found to have a role to play in \cite{markard2016socio} can be considered. Similarly, not all of the Swiss parliament can be reflected within this study. All actors are therefore aggregated down to a size of roughly ten actors in total. 

\paragraph{The actor preferred states} The actor preferred states are given in \autoref{tab:preferredStates}. They are identical to the preferred states of their respective electorate at that point in time. Their causal beliefs are given in \autoref{tab:causalBeliefs}. The causal beliefs are equivalent to the ones used within the model. This assumes that the agents have a perfect understanding of the inner workings of the system. It is not in the scope of the present research to understand the effect of an imperfect understanding of the system by the actors and therefore, it is not studied here. This is also means that there are no negative influences on the 

\begin{table}
\begin{center}
\begin{tabular}{ |c|c|c|c|c|c|c|c|c|c|c| } 
\hline
		
		& PC1 	& PC2	& S1		& S2			& S3		& S4			& S5		\\ 
		& Eco.	& Env.	& RES	& Price		& REI	& Dom. em.	& Imp. em.	\\ \hline \hline
Aff. 1		& 0.70	& 0.53	& 0.60	& 0.75 (50)	& 0.70	& 0.80 (4m)	& 0.33 (60k)	\\ \hline
Aff. 2		& 0.65	& 0.78	& 0.75	& 0.75 (50) 	& 1.00	& 0.95 (1m)	& 0.55 (40k)	\\ 
\hline
\end{tabular}
\end{center}
\caption{Preferred states for the electorate agents in both affiliation on a the interval [0,1].}
\label{tab:preferredStates}
\end{table}

%\item S1: renewable energy production
%\item S2: electricity prices
%\item S3: renewable energy investment level
%\item S4: domestic level emissions
%\item S5: imported emissions
%\item PC1: economy [0, 1]. % init 0.57 PC1 = 0.75 * S2 + 0.25 * S3
%\item PC2: environment [0, 1]. % init 0.392/0.393 PC2 = 0.25 * S1 + 0.25 * S3 + 0.25 * S4 + 0.25 * S5

\begin{table}
\begin{center}
\begin{tabular}{ |c|c|c|}
 \hline
 	& PC1	& PC2		\\ \hline \hline
-S1 	& 0.00	& 0.25		\\ \hline
-S2 	& 0.75	& 0.00		\\ \hline
-S3 	& 0.25	& 0.25		\\ \hline
-S4 	& 0.00	& 0.25		\\ \hline
-S5 	& 0.00	& 0.25		\\ 
 \hline
\end{tabular}
\end{center}
\caption{Causal beliefs for the agents of both affiliations. These causal relations can be read as: the impact of S1 on PC2 is 0.25. They are all given on the interval [-1,1].}
\label{tab:causalBeliefs}
\end{table}

The causal beliefs between deep core and policy core beliefs are not present in \autoref{tab:causalBeliefs} as no deep core belief is considered for this specific case.

%%%%%%%%%%%% end of the subsection

%%%%%%%%%%%% 
\subsection{The policy process model}


%%%%%%%%%%%% end of the subsection


%%%%%%%%%%%%
\subsection{The hybrid model}

The model simulation last 27 years in total with a warmup time of three years. This considers a start of year of 2016, therefore the model runs until 2043. The interval between policy process is 3 years with the policy process model being called 9 times. The scenarios are therefore triggered at time of 9 years ($t_1 = 2025$) and 18 years ($t_2 = 2034$). The evaluation interval, that is the amount of time that is used to test the effectiveness of policies is set at 3 years, similar to the interval between which policy processes are called. (At the moment an interval of ten years is also tested to observe potential consequences of such changes).

%%%%%%%%%%%% end of the subsection


%%%%%%%%%%%%%%%%%%%%%%%%%%%%%%%%%%%%%%%%%%%%%%%%%%%%%%%%%%%%%%%%%%
\section{Results}
\label{sec:}
%%%%%%%%%%%%%%%%%%%%%%%%%%%%%%%%%%%

\textcolor{red}{This section needs to be adjusted for the ACF+PL, +Co, +PK, +PI.}

%%%%%%%%%%%%%%%%%%%%%%%
\bibliographystyle{apalike} 
\bibliography{references}

\appendix



\end{document}
